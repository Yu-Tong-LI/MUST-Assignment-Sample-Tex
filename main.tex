\documentclass{must-assignment}

\usepackage{lipsum}
\usepackage{graphicx}
\usepackage{indentfirst} 
\usepackage[sort]{gbt7714}
\usepackage{amsmath,amssymb, amsfonts, amsthm}

%algorithm
\usepackage{algorithmic}
\usepackage{diagbox}
\usepackage[ruled,vlined,linesnumbered]{algorithm2e}

%hyperref
\usepackage{color,hyperref}
\definecolor{darkblue}{rgb}{0.0,0.0,0.6}
\hypersetup{colorlinks,breaklinks,
            linkcolor=blue,urlcolor=blue,
            anchorcolor=blue,citecolor=blue}

%python
\usepackage{listings}
\usepackage{color}

\definecolor{dkgreen}{rgb}{0,0.6,0}
\definecolor{gray}{rgb}{0.5,0.5,0.5}
\definecolor{mauve}{rgb}{0.58,0,0.82}

\lstset{frame=tb,
  language=Python,
  aboveskip=3mm,
  belowskip=3mm,
  showstringspaces=false,
  columns=flexible,
  basicstyle={\small\ttfamily},
  numbers=none,
  numberstyle=\tiny\color{gray},
  keywordstyle=\color{blue},
  commentstyle=\color{dkgreen},
  stringstyle=\color{mauve},
  breaklines=true,
  breakatwhitespace=true,
  tabsize=3
}

\newcommand*{\name}{xxxxxxxxxxxx}
\newcommand*{\id}{xxxxxxxxxx}
\newcommand*{\email}{\href{mailto:xxxxxxxx}{xxxxxxxxxxx}}
\newcommand*{\course}{XXXXXXXXXX}
\newcommand*{\assignment}{Assignment}
\newcommand*{\timetable}{December 24, 2022}

%begin
\begin{document}

\maketitle
\setlength{\parindent}{2em}
%\setlength{\parskip}{1em}
%\begin{abstract}
%.........    
%\end{abstract}
\section{Question 1}
xxxxx, xxxxx \cite{1}
\subsection{XXXX}
xxxxx, xxxxx \cite{2,3,4}

\section{Question 2}
xxxxx, xxxxx

An example of the Algorithm 1.


\begin{algorithm} 
	\SetAlgoVlined 
	\caption{Control policy construction} 
	\KwIn{Control parameter $r_i$, time series $Backgrd(T_i)$=${T_1,T_2,\ldots, T_n}$ and similarity threshold $\theta_r$} 
	\KwOut{Control policy $con(r_i)$} 
	$con(r_i)= \Phi$\; 
	\For{$j=1; j \le n; j \ne i$} 
	{ 
		float $maxSim=0$\; 
		$r^{maxSim}=null$\; 
		\While{not end of $T_j$} 
		{ 
			compute Jaro($r_i,r_m$);~~ \% here are the commentary texts 

		} 
		$con(r_i)=con(r_i)\cup {r^{maxSim}}$\; 
	} 
	return $con(r_i)$\; 
\end{algorithm}

\section{Question 3}
xxxxx, xxxxx

Formal expression is very important.

Example 1:
\begin{equation}
\label{eq1}
	e^{\pi i}+1=0
\end{equation}

Example 2:
\begin{equation}
\label{eq2}
	a^2+b^2=c^2
\end{equation}

If no equation number is needed, we can use double dollars at the beginning and end of the equation.

$$
\cos{x}+\sin{y}=1.
$$

Example 3:
\begin{equation}
\label{eq3}
\quad\dbinom{n}{m}=\dbinom{n}{n-m}=C_n^m=C_n^{n-m}
\end{equation}

Example 4:
\begin{equation}
    (a + b)^3 = (a + b) (a + b)^2=a^3 + 3a^2b + 3ab^2 + b^3     
\end{equation}

Here are more examples of mathematics equations or expression.


\begin{equation}
  x = a_0 + \cfrac{1}{a_1 
          + \cfrac{1}{a_2 
          + \cfrac{1}{a_3 + \cfrac{1}{a_4} } } }
\end{equation}




\begin{equation*}
\frac{
    \begin{array}[b]{r}
      \left( x_1 x_2 \right)\\
      \times \left( x'_1 x'_2 \right)
    \end{array}
  }{
    \left( y_1y_2y_3y_4 \right)
  }
\end{equation*}


\[
P\left(A=2\middle|\frac{A^2}{B}>4\right)
\]


\[
M = \begin{bmatrix}
       \frac{5}{6} & \frac{1}{6} & 0           \\[0.3em]
       \frac{5}{6} & 0           & \frac{1}{6} \\[0.3em]
       0           & \frac{5}{6} & \frac{1}{6}
     \end{bmatrix}
\]


\[
M = \bordermatrix{~ & x & y \cr
                  A & 1 & 0 \cr
                  B & 0 & 1 \cr}
\]


\[ f(n) =
  \begin{cases}
    n/2       & \quad \text{if } n \text{ is even}\\
    -(n+1)/2  & \quad \text{if } n \text{ is odd}
  \end{cases}
\]



\[
\left(
    \begin{array}{c}
      n \\
      r
    \end{array}
  \right) = \frac{n!}{r!(n-r)!}
\]



Here are some logic expressions:


$$
(\forall s\in\overline{K})(\forall\sigma\in\Sigma)(\forall s^\prime\in\overline{K})s\sigma\in L(G)~\&~s^\prime\sigma\in L(G)~\&~Ps=Ps^\prime\implies s^
\prime\in\overline{K}.
$$




For more details about mathematics equations or expressions, see \url{https://en.wikibooks.org/wiki/LaTeX/Mathematics}.


%\newpage
\newpage
\bibliographystyle{gbt7714-numerical}
  \bibliography{sample}

%end
\end{document}
